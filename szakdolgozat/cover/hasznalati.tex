\pagestyle{empty}

\noindent \textbf{\Large CD Használati útmutató}

\vskip 1cm
A CD-n található forráskód az ImageProcessing mappában van. Onnan lehet futtatni a kódokat is. Ehhez először létre kell hozni a virtuális környezetet, ami pedig a szakdoga nevű mappában található.
A fő kód az ./ImageProcessing/src mappában található meg image\_process.ipynb néven, ehhez szükséges lesz a virtuális környezet aktiválása. Emellett kell egy python is arra a számítógépre, ahol futtatva lesz.
Az ./ImageProcessing/samples almappában található 2 mappa images valaminet recognized\_texts néven. Az images mappában szereplő fájlok a bemeneti képeket szolgáltatták, emellett a ./recognized\_texts/images mappa pedig azokat a kimeneti képeket, amik a képfeldolgozó program kiad akkordonként egy kottából.
És végül az ./ImageProcessing/Examples mappában találhatóak azok a kódok, amik a szerializálást mutatják be, van hogy konzolos kimenettel, van amelyik gráfos kimenettel szolgál, és itt van az az rdf\_query.ipynb ami a SPARQL-es lekérdezést valósítja meg. Emellett itt találhatóak a programok által felhasznált XML források és az egyikhez megadott XMLSchema is.