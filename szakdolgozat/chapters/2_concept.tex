\Chapter{Koncepció}

\Section{RDF bemutatása}

Subject – predicate  - object

Az RDF az egy interneten levő adatokhoz tartozó standard modell. Lehetővé teszi, hogy adat összefésülést hajtson végre még abban az esetben is, mikor a séma alatt szereplő adatok hiányosak, és kifejezetten támogatja a séma fejlődését az adatok megváltoztatása nélkül.
Az RDF az URI-val nevezi meg azt a kapcsolatot, amivel a link működik az interneten, hogy egyik pontból a másikba elnavigáljon. (Ezt nevezzük hármasoknak) Ezzel az egyszerű modellel lehetővé válik az, hogy struktúrált vagy félig struktúrált adathalmazok is keverhetővé, közzétehetővé és megoszthatóvá váljanak.

Ez a kapcsolati struktúra úgy működik, hogy a két forrás adja a csomópontokat, és az él pedig a kapcsolat nevét. Lényegében ez egy gráf, a csomópont a forrás az él pedig a predikátum, így ha két webhelyet szeretnék összekötni, akkor a kiinduló webhely lesz a téma(subject) a link megnevezés, ami átirányítja a másik csomópontba, az lesz a predikátum(predicate) és ahova érkezik pedig az objektum(object).


\Section{OCR bemutatása}

A fejezet tartalma témától függően változhat. Az alábbiakat attól függően különböző arányban tartalmazhatják.
\begin{itemize}
\item Irodalomkutatás. Amennyiben a dolgozat egy módszer kidolgozására, kifejlesztésére irányul, akkor itt lehet részletesen végignézni (módszertani vagy időrendi bontásban), hogy az eddigiekben milyen eredmények születtek a témakörben.
\item Technológia. Mivel jellemzően kutatásról vagy szoftverfejlesztésről van szó, ezért annak a jellemző elemeit, technikai részleteit itt kell bemutatni.
Ez tehát egy módszeres bevezetés ahhoz, hogy ha valaki nem jártas a témakörben, akkor tudja, hogy a dolgozat milyen aktuálisan elérhető eredményeket, eszközöket használt fel.
\item Piackutatás. Bizonyos témáknál új termék vagy szolgáltatás kifejlesztése a cél.
Ekkor érdemes annak alaposan utánanézni, hogy aktuálisan milyen eszközök érhetők el a piacon.
Ez szoftverek esetében a hasonló alkalmazások bemutatását, táblázatos formában történő összehasonlítását jelentheti.
Szerepelhetnek képek és észrevételek a viszonyításként bemutatott alkalmazásokhoz.
\item Követelmény specifikáció. Külön szakaszban érdemes részletesen kitérni az elkészítendő alkalmazással kapcsolatos követelményekre.
Ehhez tartozhatnak forgatókönyvek (\textit{scenario}-k).
A szemléletesség kedvéért lehet hozzájuk képernyőkép vázlatokat is készíteni, vagy a használati eseteket más módon szemléltetni.
\end{itemize}

\Section{Amit csak említés szintjén érdemes szerepeltetni}

Az olvasóról annyit feltételezhetünk, hogy programozásban valamilyen szinten járatos, és a matematikai alapfogalmakkal sem ebben a dolgozatban kell megismertetni.
A speciális eszközök, programozási nyelvek, matematikai módszerekk és jelölések persze jó, hogy ha említésre kerülnek, de nem kell nagyon belemenni a közismertnek tekinthető dolgokba.
